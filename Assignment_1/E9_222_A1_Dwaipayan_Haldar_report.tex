\documentclass[12pt,a4paper,onecolumn]{exam}
\usepackage{amsmath}
\usepackage{amssymb}
\usepackage{graphicx}
\usepackage{float}
\usepackage{geometry}
\usepackage{tikz}
\usepackage[skins]{tcolorbox} % Use [skins] to get the rounded shape

% --- Define our simple \questionheader command ---
\newcommand{\questionheader}[1]{%
  \begin{tcolorbox}[
    enhanced,
    colback=black,
    coltext=white,
    boxrule=0pt,              
    fontupper=\Large\bfseries, 
    arc=4mm                   
  ]
  #1 
  \end{tcolorbox}%
}
% --- End of definition ---
\newtcolorbox{answerbox}[1]{
  boxrule=0.4pt,   % Sets the thickness of the border
  colback=white,   % Sets the background color
  height=#1,       % <-- Use the first argument (#1) as the height
}


\usepackage[font = small]{caption}
\usepackage{subcaption}

\newenvironment{blanksolution}
  {%
    \renewcommand{\solutiontitle}{\noindent}%
    \begin{solution}%
  }%
  {\end{solution}}  
\usepackage{listings}

\lstset{
    language=Python,
    basicstyle=\ttfamily, 
    }

\begin{document}

\begingroup  
    \centering
    \LARGE E9 222 Signal Processing in Practise\\
    \LARGE Assignment 1\\[0.5em]
    \large \today\\[0.5em]
    \large Dwaipayan Haldar\par
\endgroup
\noindent\rule{\textwidth}{0.5pt}
\printanswers
\renewcommand{\solutiontitle}{\noindent\textbf{Ans:}\enspace}


%****Question 1****

\questionheader{1. Discrete Convolution Implementation}

\begin{solution}
I implemented the algorithm using zero padding using a single loop. The results are a perfect match. Experimented with other input \verb|x| and \verb|h|. It works perfectly fine. 
\begin{verbatim}
Input x: [1 2 3 4 5]
Filter h: [ 1 -1  2]
------------------------------
Mode 'full':

  Manual:   [ 1.  1.  3.  5.  7.  3. 10.]
  Built-in: [ 1  1  3  5  7  3 10]
  Result:   MATCH
Mode 'same':

  Manual:   [1. 3. 5. 7. 3.]
  Built-in: [1 3 5 7 3]
  Result:   MATCH
Mode 'valid':

  Manual:   [3. 5. 7.]
  Built-in: [3 5 7]
  Result:   MATCH
------------------------------
\end{verbatim}

In the next step, the convolution is also performed via the topelitz matrix and circulant matrix is used to calculate and then compare the results with the fft-based circular convolution. The topelitz matrix is implemented using the \verb|topelitz| function and circulant matrix is implemented via the \verb|circulant| function. 

\begin{verbatim}
Standard Convolution Output: [1 1 3 5 2 8]

Toeplitz Convolution Matrix (H):
[[ 1.  0.  0.  0.]
 [-1.  1.  0.  0.]
 [ 2. -1.  1.  0.]
 [ 0.  2. -1.  1.]
 [ 0.  0.  2. -1.]
 [ 0.  0.  0.  2.]]

Matrix Multiplication Output: [1. 1. 3. 5. 2. 8.]
Circulant Matrix C:
[[ 1  0 -1  0]
 [ 0  1  0 -1]
 [-1  0  1  0]
 [ 0 -1  0  1]]

Matrix Result: [-2 -2  2  2]
FFT Result: [-2. -2.  2.  2.]

\end{verbatim}
\end{solution}

%****Question 2****
\questionheader{2. 1D Deconvolution}

\begin{solution}

    \begin{itemize}
        \item[1] In the first part, Wiener Filter is implemented using the given formula where K and kernel is passed as hyperparameter. 
        \item[2] In the second part, choice of K for wiener filter is important. The value of K is tuned by $(SNR)^-1$. The signal power is calculated using the variance of the observed signal. Since the audio signal is almost 0 mean, so it is a good approximation of energy. The noise level is known, so it is just the noise level squared.
    \end{itemize}
    
\end{solution}

        \begin{figure}
            \centering
            \includegraphics[scale = 0.4]{P02.png}
            \caption{Waveform of clean speech, noisy speech, naive inverse filter and wiener filter reconstruction}
            \label{fig:1}
        \end{figure}

\begin{solution}
    Fig.\ref{fig:1} gives the waveform of the clean speech, reverberant+noisy speech, inverse filter restoration and wiener filter restoration image. Inverse filter restoration amplify the noise which is expected. Wiener filter does a better job at filtering given the type of filter with which noise is incorporated.
\end{solution}

%****Question 3****
\questionheader{3. 2D Image Degradation Problem }

\begin{solution}
    Image was convolved with a motion blur signal and gaussian noise of 0.01 intensity is added to it. The motion blur routine was already given in the code.
\end{solution}


%****Question 4 & 5****
\questionheader{4 \& 5. Inverse Filtering (2D) and Wiener Filtering (2D)}

\begin{solution}
    Both the inverse filter and wiener filter reconstruction is tried on the images. SSIM and PSNR for the images is reported. As the value of K drops, there are artifacts in the restored images. Artifacts or ringing in the wiener filter is due to the fact that wiener has the assumption of stationarity, which is violated in the images specially at the edges. So, there are visible artifacts. The inverse filter has a very poor response as expected with a negative PSNR of around -25 dB and SSIM of around 0.003. Results for different cases are given below. 
\end{solution}

        \begin{figure}[H]
            \centering
            \includegraphics[scale = 0.5]{P05_1e-4.png}
            \caption{Wiener filter with K = 0.0001, SSIM = 0.0342, PSNR = 12.3348}
            \label{fig:2}
        \end{figure}
        \begin{figure}[H]
            \centering
            \includegraphics[scale = 0.45]{P05_1e-3.png}
            \caption{Wiener filter with K = 0.001, SSIM = 0.0943, PSNR = 15.1668}
            \label{fig:3}
        \end{figure}
        \begin{figure}[H]
            \centering
            \includegraphics[scale = 0.45]{P05_1e-2.png}
            \caption{Wiener filter with K = 0.01, SSIM = 0.2598, PSNR = 16.5539}
            \label{fig:4}
        \end{figure}

%****Question 6****
\questionheader{6. Parameter Estimation}

\begin{solution}
    Value of sigma, kernel size and K value in Wiener filter is swept in a range of values and the best output is taken in this case. The best output comes with kernel size = 3, sigma = 0.5 and K = 0.02. The PSNR comes out to be 17.16 dB. Visually, the output is pretty poor. That is expected as wiener gives the best output with the prior filter with which it is blurred. In this case, the image was blurred with motion filter but the recovery attempt was made with gaussian filter. That is the cause of its poor output. Fig.\ref{fig:5} gives the output
\end{solution}

        \begin{figure}
            \centering
            \includegraphics[scale = 0.4]{P06.png}
            \caption{Original Image of page.png, Blurred image and restored image}
            \label{fig:5}
        \end{figure}











\end{document}